%%-------------------Emacs PostScript "pretty-print" page width (97 columns)-------------------%%
%% This file loads a number of standard packages.  Also included at the end of the file are a number of macros.
%% Math packages-----------------%-------------------------------
\usepackage{amsmath}
%\usepackage{amssymb,mathrsfs}%provides symbols, script text in math mode (conflicts with mathdesign)
%\usepackage{bm}% bold math

%% Text packages-----------------%-------------------------------
\usepackage{textcomp,lmodern}%for special symbols 
%\usepackage[normalem]{ulem}%for strikethroughs
\usepackage[utf8]{inputenc}
\usepackage[T1]{fontenc}%modern font encoding

%% The default font is Latin Modern (replaces Computer Modern).  Following are a number of fonts 
%  with extensive math and symbol support.  To use a particular font, uncomment the associated
%  line.  If mathdesign fonts are used, the amssymb and mathrsfs packages should not be loaded
%  independently.

%% PostScript Fonts--------------%-------------------------------
%\usepackage[osf]{mathpazo}		%URW Palladio (A Palatino/Book Antiqua clone), an old style serif typeface
%\usepackage{mathptmx}				%URW Nimbus Roman (A Times clone), a transitional serif typefaces
%\usepackage[scaled]{helvet}	%URW Nimbus Sans (A Helvetica clone), a sans-serif typeface 
%\renewcommand*\familydefault{\sfdefault} %% Only if the base font of the document is to be sans serif

%% Math Design Fonts-------------%-------------------------------
%\usepackage[urw-garamond]{mathdesign}			%Old style serif typeface
\usepackage[adobe-utopia]{mathdesign}			%Transitional serif typefaces (professional)
%\usepackage[bitstream-charter]{mathdesign}	%Glyphic serif typeface optimized for low-resolution printing

%% Formatting packages-----------%-------------------------------
\usepackage{fancyvrb}%allows verbatim in footnotes
\usepackage{indentfirst}%indents first paragraph of each section

\usepackage{lineno}
\biboptions{numbers,sort&compress}%load natbib options
\usepackage{ifpdf}
\ifpdf
	\usepackage{graphicx}
	\usepackage{epstopdf}
	%Must include -enable-write18 in command line arguments (Alt+F7)
	\epstopdfDeclareGraphicsRule{.eps}{pdf}{.pdf}{ps2pdf -dEPSCrop #1 \OutputFile}
	\epstopdfsetup{suffix=}%
\else
	\usepackage{graphicx}	
	\usepackage[all,light]{draftcopy}% Places "DRAFT" on each page.  Only works with PS output.	
\fi
\graphicspath{{../Figures/BW_Figures/}{../Figures/}} %Uniquely-named B&W figures
\def\figwid{4.9in}
%\usepackage[light]{draftcopy}% Places "DRAFT" on each page.  Only works with PS output.
\usepackage{hyperref}
\hypersetup{
	pdfstartpage=1,                %Opening page number (absolute)
	pdfstartview=FitV,             %Fits the horiz. width in the window (FitV for vertical)
	%pdfstartview={XYZ null null 1},%view page at 100%
	bookmarksopen=true,            %Displays Bookmarks in the Navigation Panel
	bookmarksopenlevel=0,          %\maxdimen all levels, 0 chapters, 1 sections
	bookmarksnumbered=true,        %Numbers bookmarks with section numbers
	final=true,                    %keeps hyperref features in draft mode
	colorlinks=true,               %colors the links instead of using boxes
	urlcolor=blue,                 %makes URL hyperlinks blue (instead of pink)
	linkcolor=black,               %makes internal links black (instead of red)
	citecolor=black,               %makes citation links black (instead of green)
}

\usepackage{fmtcount,datetime} %provides commands for displaying times and dates
\usepackage{setspace} %required for double spacing and used to control vertical spacing 
\usepackage[numbers,sort&compress]{natbib}
\usepackage{fancyhdr} %required for non-standard (fancy) headers

%% The following two lines may be required for proper formating of the Table of Figures
%\usepackage{tocloft}
%\setlength{\cftfignumwidth}{3em}%increases spacing after figure numbers in Table of Figures

\usepackage{color}

\InputIfFileExists{More_Files/autotext}{}{}

\renewcommand{\thefootnote}{\fnsymbol{footnote}} %sets footnote marks to symbols

\providecommand{\Chapter}[1]{\chapter[\texorpdfstring{\MakeUppercase{#1}}{#1}]{#1}}
