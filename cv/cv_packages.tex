%% Formatting packages-----------%-------------------------------
\providecommand{\printname}{Jonathan C. Lighthall}
\newcommand{\shortname}{J.\ C.\ Lighthall}

\def\indent{3in}
%\setlength{\parindent}{0in}
%% Set indentation here
%% The letter should either be indented (each paragraph, address, and signature) or not at all
\newif\ifblock
\blockfalse
\ifblock
%% Block format
\def\indent{0in}
\setlength{\parindent}{0in}
\setlength{\parskip}{1\baselineskip}
\else
%% Indented
\def\indent{3in}
\fi

\providecommand{\printname}{Jonathan C.\ Lighthall}
\providecommand{\shortname}{J.\ C.\ Lighthall}
\providecommand{\doctitle}{From the desk of Dr.\ \shortname}

\hypersetup{pdfauthor=\printname,
            pdftitle=\doctitle, 
						}
\newif\ifprint
\ifprint
\hypersetup{urlcolor=black}
\else
\hypersetup{urlcolor=blue}
\fi

\newcommand{\addressprefix}{}

\newcommand{\homeaddress}{%
%816 S. Adams St. Apt. C305, Westmont, IL 60559-3606\\%
%\href{mailto:jonathan.c.lighthall@wmich.edu}{jonathan.c.lighthall@wmich.edu}\\%
%(248) 797-3674\\%
13--6395 Hawthorn Lane\\ Vancouver, BC\hspace{6pt}V6T~OC1
%\\Canada
}

\newcommand{\addresssuffix}{\\
office: (604) 222-1047 ext. 6859
| cell: (778) 999-6006\\
\href{mailto:lighthall@triumf.ca}{lighthall@triumf.ca}\\%
}

\newcommand{\workaddress}{
\begin{tabular}{r|p{135pt}}
	Western Michigan University&Argonne National Laboratory\\
	Physics Department&Physics Division\\
	1903 W. Michigan Ave.&9700 S. Cass Ave.\\
	Kalamazoo, MI 49008-5252&Argonne, IL 60439\\
	\href{mailto:jonathan.c.lighthall@wmich.edu}{jonathan.c.lighthall@wmich.edu}&(630) 252-4803\\
\end{tabular}
}

\renewcommand{\workaddress}{TRIUMF\\
4004 Wesbrook Mall, 
Vancouver, BC\hspace{6pt}V6T~2A3,
Canada}

%% Set margin for sender address
\renewenvironment{changemargin}[1]{%
  \begin{list}{}{%
    \setlength{\topsep}{0pt}%
    \setlength{\leftmargin}{#1}%
		\addtolength{\textwidth}{-#1}
    %\setlength{\rightmargin}{#1}%
     }\item[]}{\end{list}}

\newcommand{\nameaddress}[1]{
\textrm{\printname}\\
\addressprefix
#1%
%\homeaddress
%\workaddress
\addresssuffix
}

\newcommand{\recip}{Recipient}
\newcommand{\recipaddress}{Address\\City\\Country}
\newcommand{\comp}{Competition}

\newcommand{\ltrwrapper}[2] {%one argument, text
\clearpage
\pagestyle{empty}
\begin{changemargin}{\indent}
\nameaddress{#1}%\\
\ifblock

\else
\\
\fi
\displaydate{letterdate}
\end{changemargin}
\ifblock

\else
\vspace*{1.0\baselineskip}%triple space
\fi
\noindent \recip{} \comp{}\\
\recipaddress{}
\ifblock

\else

\vspace*{1.0\baselineskip}%triple space
\fi

\noindent
To Whom It May Concern at \recip,%
%Dear \recip,
\ifblock

\else
\\
\fi

#2%
\ifblock

\else
\\
\fi
\begin{changemargin}{\indent}
Sincerely,

\vspace*{-0.9\baselineskip}
\begin{figure}[h!]
\begin{changemargin}{\indent}
		%\includegraphics[height=1cm,keepaspectratio]{../../_DS12247.eps}
		%\includegraphics[height=2.22cm,keepaspectratio]{_DS14681}%full size
		\includegraphics[height=1cm,keepaspectratio]{_DS14681}%fits in space
		%\includegraphics[height=1cm,keepaspectratio]{Holly_signature}
		\end{changemargin}
\end{figure}
\vspace*{-1.0\baselineskip}
\printname
\end{changemargin}
}

\graphicspath{{../ltr/}}

\renewcommand{\nameaddress}[1]{
\textbf{\printname}\\
%\addressprefix
#1%
%\homeaddress
%\workaddress
\addresssuffix
}

\providecommand{\authname}{\printname}
\providecommand{\doctitle}{\shortname CV}
\hypersetup{
  pdfauthor=\authname,           %Adds author name to PDF properties
  pdftitle=\doctitle,            %Adds thesis title to PDF properties
  pdfstartpage=1,                %Opening page number (absolute)
  pdfstartview={XYZ null null 1},%view page at 100%
	urlcolor=blue,                  
}

%% Bibliography packages---------%-------------------------------
\usepackage{multibib}
%% Multibib generates multiple bibliographic sections.  Each bibliography requires a separate
%% .bbl file which is generated from an .aux file by BibTex.  This can be done manually or with
%% a script.  For example, create a batch file in the current directory with the following
%% contents:
%% for %%f in (*.aux) do "C:\Program Files\MiKTeX 2.9\miktex\bin\bibtex.exe" %%~nf
%% Then, define a new output profile (in TeXnicCenter) which points to the new batch file.
%\IfFileExists{multibib.bat}{}{\errmessage{Batch file for multibib not found}}
\newcites{talk}{Presentations}
\newcites{proc}{Proceedings}
\newcites{invi}{Invited Talks}

%% Application/CV settings-------%-------------------------------
\usepackage{sectsty}
\sectionfont{\normalsize} %use 

\newcommand{\heliosonly}[1]{#1}
\newcommand{\jclonly}[1]{#1}

\newcommand{\desc}[1]{
\renewenvironment{quote}{%
\list{}{%
    \setlength{\leftmargin}{22pt}
    %\rightmargin0pt%\leftmargin
		\setlength{\itemsep}{0pt}
	\setlength{\parskip}{0pt}
	\setlength{\parsep}{0pt}
	\setlength{\topsep}{0pt}
	\setlength{\partopsep}{0pt}
  }
  \item\relax
}
{\endlist}
\begin{quote}
%\vspace*{-0.35\baselineskip}%
%\textit
\footnotesize
{#1}% \the\leftmargin, \the\rightmargin
%\vspace*{-0.35\baselineskip}%
\end{quote}
}

\renewenvironment{itemize}%
{ \begin{list}%
   {\textbullet}%
   {%
	%\setlength{\topsep}{0pt}
	%\setlength{\partopsep}{0pt}
  \setlength{\itemsep}{0pt}
  \setlength{\parskip}{0pt}
  \setlength{\parsep}{0pt}
		}
		}%
{ \end{list} }